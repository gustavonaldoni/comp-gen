\documentclass{article}
% Pacotes Essenciais ========================================================
\usepackage[utf8]{inputenc} % Acentuação
\usepackage[T1]{fontenc}
\usepackage{graphicx,xcolor} % Gráficos e Cores
\usepackage[lmargin=3cm,tmargin=3cm,rmargin=2cm,bmargin=2cm,headheight=15.13202pt]{geometry}    % Margens
\usepackage[onehalfspacing]{setspace} % Espaçamento de Um e Meio
\usepackage{indentfirst} % Indentar o primeiro
\usepackage{enumerate} % Enumeração
\usepackage{amsfonts}
\usepackage{amsmath}
\usepackage{amssymb}
\usepackage{amsthm}
\usepackage{hyperref}
\usepackage{courier}

\usepackage{mathptmx}

% Embelezamento =============================================================
\usepackage{fancyhdr} % Fancy é para deixar bonito

% Pacotes Adicionais ========================================================
\usepackage{blindtext,comment} % Texto cego e comentário
\usepackage{multicol,multirow} % Multicolunas e multitabelas

\title{Exercises about Computer Science}
\author{Gustavo Azevedo Naldoni}
\date{\today}
\begin{document}
\input{page_style.tex}
\maketitle
\section{Exercises}
\begin{enumerate}
\item What will be the output of the following code?

            \begin{verbatim}
#include<stdio.h>

int main()
{
    char *str;
    str = "%s";
    printf(str, "K\n");
    return 0;
}
            \end{verbatim}
    
            \begin{enumerate}
                \item Error.
                \item No output.
                \item K.
                \item \%s
                \item \begin{verbatim}K \n \end{verbatim}
            \end{enumerate}
\item What is the output of the following Java code?

            \begin{verbatim}
public class array
{
	public static void main(String args)
	{
		int arr = {1,2,3,4,5};
		System.out.println(arr2);
		System.out.println(arr4);
	}
}
            \end{verbatim}
    
            \begin{enumerate}
                \item 4 and 2.
                \item 2 and 4.
                \item 5 and 3.
                \item 3 and 5.
                \item 4 and 3.
            \end{enumerate}
\item On C programming there is a common used structure defined as \texttt{(void *) 0}. What is it?
    
            \begin{enumerate}
                \item The NULL pointer.
                \item The void pointer.
                \item Error.
                \item Garbage value stored on RAM.
                \item Garbage value stored on disk.
            \end{enumerate}
\item When dealing with an empty stack \texttt{s}, what sequence of operations gives the result string \texttt{cat}?
            
            \begin{enumerate}
                \item \texttt{push(c, s); push(a, s); push(t, s); pop(s); pop(s); pop(s);}
                \item \texttt{push(c,s); pop(s); push(a,s); pop(s); push(t,s); pop(s);}
                \item \texttt{pop(c); pop(a); pop(t);}
                \item \texttt{push(c, s); push(a, s); pop(t);}
                \item \texttt{push(a); push(a, s); push(t, s); pop(a); pop(s); pop(s);}
            \end{enumerate}
\item What will be the output of the program if the size of pointer is 4-bytes?

            \begin{verbatim}
#include<stdio.h>

int main()
{
    printf("%d, %d\n", sizeof(NULL), sizeof(""));
    return 0;
}
            \end{verbatim}
    
            \begin{enumerate}
                \item 2, 1.
                \item 1, 2.
                \item 2, 2.
                \item 4, 1.
                \item 4, 2.
            \end{enumerate}
\end{enumerate}
\section{Answers}
\begin{enumerate}
\item C
\item D
\item A
\item B
\item D
\end{enumerate}
\end{document}
