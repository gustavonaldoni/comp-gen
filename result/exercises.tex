\documentclass{article}
\usepackage{graphicx}
\usepackage{enumerate}
\usepackage{float}
\usepackage{multicol}
\usepackage{hyperref}

\usepackage{amsmath, amsfonts, amssymb}

\usepackage{fancyhdr}

\title{Exercises about Computer Science}
\author{Gustavo Azevedo Naldoni}
\date{\today}
\begin{document}
\pagestyle{fancy}

\fancyhead{} % clear all header fields
\fancyhead[RO,LE]{Exercises about Computer Science}

\fancyfoot{} % clear all footer fields
\fancyfoot[LE,RO]{Page \thepage}
\fancyfoot[LO,CE]{\copyright \hspace{0.10cm} Gustavo Azevedo Naldoni}
\maketitle
\section{Exercises}
\begin{enumerate}
\item Encontre a derivada das seguintes funções, utilizando a \emph{Regra da Cadeia}:

    \begin{enumerate}
        \item $f(x) = \sqrt{5x + 1}$
        \item $g(\theta) = \cos^2 \theta$
        \item $y = e^{\tan (\theta)}$
        \item $f(t) = t \sin (\pi t)$
        \item $y = x^x$
    \end{enumerate}
\item An OR gate has 6 inputs. How many input words are in its truth table?
    
            \begin{enumerate}
                \item 64
                \item 32
                \item 16
                \item 128
                \item None of the above
            \end{enumerate}
\item What is the 2's-complement representation of -24 in a 16-bit microcomputer?
    
            \begin{enumerate}
                \item 0000 0000 0001 1000
                \item 1111 1111 1110 0111
                \item 1111 1111 1110 1000
                \item 0001 0001 1111 0011
                \item None of the above
            \end{enumerate}
\item Which gate is known as a universal gate?
    
            \begin{enumerate}
                \item NOT gate
                \item AND gate
                \item NAND gate
                \item XOR gate
                \item None of the above
            \end{enumerate}
\item What is the abbreviation of "binary digit"?
    
            \begin{enumerate}
                \item 0 and 1.
                \item Byte.
                \item Bit.
                \item Bin.
                \item Base.
            \end{enumerate}
\item On C programming there is a common used structure defined as \texttt{(void *) 0}. What is it?
    
            \begin{enumerate}
                \item The NULL pointer.
                \item The void pointer.
                \item Error.
                \item Garbage value stored on RAM.
                \item Garbage value stored on disk.
            \end{enumerate}
\item If a variable is a pointer to a structure, then which of the following operator is used to access data members of the structure through the pointer variable?
    
            \begin{enumerate}
                \item .
                \item \%
                \item \&
                \item *
                \item ->
            \end{enumerate}
\item What will be the output of the following code?

            \begin{verbatim}
#include<stdio.h>

int main()
{
    char str20 = "Hello";
    char *const p=str;
    *p='M';
    printf("%s\n", str);
    return 0;
}
            \end{verbatim}
    
            \begin{enumerate}
                \item Hello
                \item Mello
                \item HMello
                \item MHello
                \item Mehllo
            \end{enumerate}

    
\item What will be the output of the following code?

            \begin{verbatim}
#include<stdio.h>

int main()
{
    char *str;
    str = "%s";
    printf(str, "K\n");
    return 0;
}
            \end{verbatim}
    
            \begin{enumerate}
                \item Error.
                \item No output.
                \item K.
                \item \%s
                \item \begin{verbatim}K \n \end{verbatim}
            \end{enumerate}
\item What will be the output of the program if the size of pointer is 4-bytes?

            \begin{verbatim}
#include<stdio.h>

int main()
{
    printf("%d, %d\n", sizeof(NULL), sizeof(""));
    return 0;
}
            \end{verbatim}
    
            \begin{enumerate}
                \item 2, 1.
                \item 1, 2.
                \item 2, 2.
                \item 4, 1.
                \item 4, 2.
            \end{enumerate}
\end{enumerate}
\section{Answers}
\begin{enumerate}
\item \begin{enumerate}
        \item $f'(x) = \displaystyle \frac{5}{2 \sqrt{5x + 1}}$
        \item $g'(\theta) = -2 \sin (\theta) \cos (\theta)$
        \item $y' = e^{\tan (\theta)} \sec ^2 (\theta)$
        \item $f'(t) = \sin (\pi t) + \pi t \cos (\pi t)$
        \item $y' = x^x (\ln (x) + 1)$
    \end{enumerate}
\item A
\item C
\item C
\item D
\item A
\item E
\item B
\item C
\item D
\end{enumerate}
\end{document}
